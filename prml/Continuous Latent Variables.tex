 \chapter{Principal Component Analysis}
 \section{Introduction}
 Principal Component Analysis is widely used for applications such as dimensionality reduction,lossy data compression,feature extraction,and data visualization.Also known as the Karhunen-Loeve transform.There are two definitions giving rise to the same algorithm.PCA can be defined as the orthogonal projection of the data onto a lower dimensional linear space,known as the principal subspace,such that the variance of the projected data  is maximized.Equivalently,it can be defined as the linear projection the minimizes the average projection cost,
defined as  the linear projection that minimizes the average projection cost,defined as the mean squared distance between the data points and their projections.
\section{Maximum variance formulation}
Consider a data set of observations $\{x_n\}$ where $n = 1,...,N$,and $x_n$ is a Euclidean variable with dimensionality D.
Our goal is to project the data onto a space having dimensionality $M < D$ while maximizing the variance of the projected
data.We define the direction of this space using a D-dimensional unit vector $\mathbf{u_1^T}\mathbf{u_1} = 1$.Each 
data point $\mathbf{x_n}$ is then projected onto a scalar value $\mathbf{u_1^T}\mathbf{x_n}$.The mean of the projected 
data is $\mathbf{u_1^T}\bar{\mathbf{x}}$ where the $\bar{\mathbf{x}}$ is the sample set mean given by
\begin{align}
\bar{\mathbf{x}} = \frac{1}{N}\sum_{n=1}^{N}{\mathbf{x_n}}
\end{align}                                  
and the variance of the projected data is given by
\begin{align}
\frac{1}{N}\sum_{n=1}^{N}\{\mathbf{u_1^T}\mathbf{x_n} - \mathbf{u_1^T}\bar{\mathbf{x}}\}^2 
&= \frac{1}{N}\sum_{n=1}^{N}{\{\mathbf{u_1^T}(\mathbf{x_n} - \bar{\mathbf{x}})\}^2} \\
&= \frac{1}{N}\sum_{n=1}^{N}{\{\mathbf{u_1^T(\mathbf{x_n - \bar{\mathbf{x}}})(\mathbf{x_n -\bar{x}})^T\mathbf{u_1^T} }  \}} \\
&= \mathbf{u_1^T}\mathbf{S}\mathbf{u_1}
\end{align}
where $\mathbf{S}$ is the data covariance matrix defined by
\begin{align}
\mathbf{S} = \frac{1}{N}\sum_{n=1}^{N}(\mathbf{x_n}-\bar{\mathbf{x}})(\mathbf{x_n}-\mathbf{\bar{x}})^T
\end{align}
We now maximize the projected variance $\mathbf{u_1^T}\mathbf{S}\mathbf{u_1}$ with respect to $\mathbf{u_1}$,which is a
constrained maximization to prevent $\parallel\mathbf{u_1}\parallel\rightarrow \infty$ .The appropriate constraint 
comes from the normalization condition $\mathbf{u_1^T}\mathbf{u_1}=1$.To enforce this constraint,we introduce a 
Lagrange multiplier that we shall denote by $\lambda_1$,and then make an unconstrained maximization of
\begin{equation}
\mathbf{u_1^T}\mathbf{S}\mathbf{u_1} + \lambda_1(1-\mathbf{u_1^T}\mathbf{u_1})
\end{equation}
By setting the derivative with respect to $\mathbf{u_1}$ equal to zero,we see that this quantity will have a stationary
point when
\begin{equation}
\mathbf{S}\mathbf{u_1} = \lambda_1\mathbf{u_1}
\end{equation}
which says that $\mathbf{u_1}$ must be an eigenvector of $\mathbf{S}$.If we left-multiply by $\mathbf{u_1^T}$ and make use
of $\mathbf{u_1^T}{u_1} = 1$,we see that the variance is given by
\begin{equation}
\mathbf{u_1^TSu_1} = \lambda_1
\end{equation}
and so the variance will be a maximum when we set $\mathbf{u_1}$ equal to the eigenvector having the largest 
eigenvalue $\lambda_1$.This eigenvector is known as the first principal component.

\section{Minimum-error formulation}


\section{Applications of PCA}